\documentclass[10pt]{beamer}
\usepackage[utf8]{inputenc}
\usepackage[ngerman]{babel}
\usepackage{minted}

\usemintedstyle{manni}

\definecolor{verydarkgray}{RGB}{15,15,15}
\renewcommand{\theFancyVerbLine}{\ttfamily \textcolor[rgb]{0.87,0.87,0.87}{\small \oldstylenums{\arabic{FancyVerbLine}}}}

\usetheme{metropolis}
%\setbeamerfont{headline}{size=\large}
%\setbeamerfont*{section in head/foot}{size=\tiny}
%\setbeamertemplate{toc}{circle}
%\setbeamertemplate{itemize subitem}[triangle] % if you want a triangle
%\setbeamercovered{transparent}

%Für [Seite]/[Anzahl-Seiten] in der unteren rechten ecke
\newcommand*\oldmacro{}%
\let\oldmacro\insertshorttitle%
\renewcommand*\insertshorttitle
{
	\oldmacro\hfill%
	\thepage\,/\,\pageref{LastPage}
}

\usecolortheme{owl}

%\beamertemplatenavigationsymbolsempty

\author{Hauke Stieler}
\title{Eine Einführung in go}
\date{\footnotesize \today}
\institute{Fachbereich Informatik der Universität Hamburg}\titlegraphic{\hfill\includegraphics[width=0.65\textwidth]{images/gopher-inv}}

\begin{document}
	\maketitle
	
	%%%%%%%%%%%%%%%%%%%%%%%%%%%%%%%%%%%%%%%%%%%%%%%%%%%%%%%%%%%%%%%%%%%%%%%
	
	\begin{frame}{Agenda}
		\begin{itemize}
			\item some history
			\item basic features
			\item cool web stuff
			\item concurrency212121
			\item interfaces
		\end{itemize}
	\end{frame}

	%%%%%%%%%%%%%%%%%%%%%%%%%%%%%%%%%%%%%%%%%%%%%%%%%%%%%%%%%%%%%%%%%%%%%%%

	\begin{frame}{Why go?}
		In 2007, three guys at Google were frustrated with the existing languages for writing server software:
		\begin{itemize}
			\item Compiling C++ was too slow
			\item Writing Java felt too verbose
			\item Aversion against inheritance and design patterns
			\item Getting concurrency right was hard
		\end{itemize}
	\end{frame}

	%%%%%%%%%%%%%%%%%%%%%%%%%%%%%%%%%%%%%%%%%%%%%%%%%%%%%%%%%%%%%%%%%%%%%%%

	\begin{frame}[fragile]{C++}
		\begin{minted}[bgcolor=verydarkgray,linenos,tabsize=4]{cpp}
// Within large projects, popular header files
// get included thousands of times and hence
// have to be recompiled over and over again
#include <iostream>
#include <string>
#include <vector>
		\end{minted}
		\pause
		\vfill
		\texttt{gcc} copies specified file by \mintinline[bgcolor=verydarkgray]{cpp}{#include} recursively into source file. The \textit{same} header file gets compiled over and over again.
		\vfill
\end{frame}

	%%%%%%%%%%%%%%%%%%%%%%%%%%%%%%%%%%%%%%%%%%%%%%%%%%%%%%%%%%%%%%%%%%%%%%%

	\begin{frame}[fragile]{Java I}
		Let's do some Java.\\
		Write a \mintinline[bgcolor=verydarkgray]{java}{public class Person} that does the following:
		\begin{itemize}
			\item store a string \texttt{name}
			\item store an int \texttt{age}
		\end{itemize}
		Simple, right?
		\vfill\pause
		NO :(
\end{frame}
	
	%%%%%%%%%%%%%%%%%%%%%%%%%%%%%%%%%%%%%%%%%%%%%%%%%%%%%%%%%%%%%%%%%%%%%%%
	
	\begin{frame}[fragile]{Java I}
		\begin{minted}[bgcolor=verydarkgray,linenos,stripnl=false,tabsize=4,breaklines]{java}
public class Person {
	private String name;
	private int age;
	
	public PersonBean(String name, int age) {
		this.name = name;
		this.age = age;
	}
	
	public String getName() {
		return name;
	}
	
	public void setName(String name) {
		this.name = name;
	}
	
		\end{minted}
\end{frame}
	\begin{frame}[fragile]{Java II}
		\begin{minted}[bgcolor=verydarkgray,linenos,firstnumber=last,tabsize=4,breaklines]{java}
	public int getAge() {
		return age;
	}
	
	public void setAge(int age) {
		this.age = age;
	}
	
	@Override
	public String toString() {
		return "PersonBean [" + "name=" + name + ", " +	"age=" + age + "]";
	}
	
	@Override
	public int hashCode() {
		final int prime = 31;
		\end{minted}
\end{frame}
	\begin{frame}[fragile]{Java III}
		\begin{minted}[bgcolor=verydarkgray,linenos,firstnumber=last,tabsize=4,breaklines]{java}
		int result = 1;
		result = prime * result + age;
		result = prime * result + ((name == null) ? 0 : name.hashCode());
		return result;
	}
	
	@Override
	public boolean equals(Object obj) {
		if (this == obj)
			return true;
		if (obj == null)
			return false;
		if (getClass() != obj.getClass())
			return false;
		PersonBean other = (PersonBean) obj;
		if (age != other.age)
		\end{minted}
\end{frame}
	\begin{frame}[fragile]{Java IV}
		\vspace{-2.5cm}
		\begin{minted}[bgcolor=verydarkgray,linenos,firstnumber=last,tabsize=4,breaklines]{java}
			return false;
		if (name == null) {
			if (other.name != null)
				return false;
		} else if (!name.equals(other.name))
			return false;
		return true;
	}
}
		\end{minted}
\end{frame}
\end{document}