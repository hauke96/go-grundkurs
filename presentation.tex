\documentclass[10pt]{beamer}
\usepackage[utf8]{inputenc}
\usepackage[ngerman]{babel}
\usepackage{minted}
\usepackage{textcomp}

\setminted{bgcolor=verydarkgray,linenos,tabsize=4,breaklines}
\setminted[bash]{bgcolor=verydarkgray,linenos=false,tabsize=4,breaklines}

\renewcommand{\theFancyVerbLine}{\ttfamily \textcolor[rgb]{0.87,0.87,0.87}{\small \oldstylenums{\arabic{FancyVerbLine}}}}

% LIGHT
%\definecolor{verydarkgray}{RGB}{235,235,235}
%\usemintedstyle{manni}
%\usetheme{metropolis}
%\usecolortheme[snowy]{owl}
%\newcommand{\inv}{}

% DARK
\definecolor{verydarkgray}{RGB}{15,15,15}
\usemintedstyle{manni}
\usetheme{metropolis}
\usecolortheme{owl}
\newcommand{\inv}{-inv}

%Für [Seite]/[Anzahl-Seiten] in der unteren rechten ecke
\newcommand*\oldmacro{}%
\let\oldmacro\insertshorttitle%
\renewcommand*\insertshorttitle
{
	\oldmacro\hfill%
	\thepage\,/\,\pageref{LastPage}
}

%\beamertemplatenavigationsymbolsempty

\author{Hauke Stieler}
\title{Eine Einführung in go}
\date{\footnotesize \today}
\institute{Fachbereich Informatik der Universität Hamburg}\titlegraphic{\hfill\includegraphics[width=0.65\textwidth]{images/gopher\inv}}

\begin{document}
	\maketitle
	
	%%%%%%%%%%%%%%%%%%%%%%%%%%%%%%%%%%%%%%%%%%%%%%%%%%%%%%%%%%%%%%%%%%%%%%%
	
	\begin{frame}{Agenda}
		\begin{itemize}
			\item some history
			\item basic features
			\item cool web stuff
			\item concurrency212121
			\item interfaces
		\end{itemize}
	\end{frame}

	%%%%%%%%%%%%%%%%%%%%%%%%%%%%%%%%%%%%%%%%%%%%%%%%%%%%%%%%%%%%%%%%%%%%%%%

	\begin{frame}{Why go?}
		In 2007, three guys at Google were frustrated with the existing languages for writing server software:
		\begin{itemize}
			\item Compiling C++ was too slow
			\item Writing Java felt too verbose
			\item Aversion against inheritance and design patterns
			\item Getting concurrency right was hard
		\end{itemize}
	\end{frame}

	%%%%%%%%%%%%%%%%%%%%%%%%%%%%%%%%%%%%%%%%%%%%%%%%%%%%%%%%%%%%%%%%%%%%%%%

	\begin{frame}[fragile]{C++}
		\begin{minted}{cpp}
// Within large projects, popular header files
// get included thousands of times and hence
// have to be recompiled over and over again
#include <iostream>
#include <string>
#include <vector>
		\end{minted}
		\\\\\\
		\pause
		\texttt{gcc} copies specified file by \mintinline{cpp}{#include} recursively into source file. The \textit{same} header file gets recompiled over and over again.\\
		\hfill\\
		\textrightarrow ~Rob Pike: \href{https://www.youtube.com/watch?v=5kj5ApnhPAE}{Public Static Void} at OSCON 2010
		\vfill
\end{frame}

	\begin{frame}
		\begin{center}
			\includegraphics[width=0.65\textwidth]{images/compiling\inv}
		\end{center}
	\end{frame}

	%%%%%%%%%%%%%%%%%%%%%%%%%%%%%%%%%%%%%%%%%%%%%%%%%%%%%%%%%%%%%%%%%%%%%%%

	\begin{frame}[fragile]{Java I}
		Let's do some Java.\\
		Write a \mintinline{java}{public class Person} that does the following:
		\begin{itemize}
			\item store a string \texttt{name}
			\item store an int \texttt{age}
		\end{itemize}
		Simple, right?
		\vfill\pause
		NO :(
\end{frame}
	
	%%%%%%%%%%%%%%%%%%%%%%%%%%%%%%%%%%%%%%%%%%%%%%%%%%%%%%%%%%%%%%%%%%%%%%%
	
	\begin{frame}[fragile]{Java I}
		\begin{minted}[stripnl=false]{java}
public class Person {
	private String name;
	private int age;
	
	public PersonBean(String name, int age) {
		this.name = name;
		this.age = age;
	}
	
	public String getName() {
		return name;
	}
	
	public void setName(String name) {
		this.name = name;
	}
	
		\end{minted}
\end{frame}
	\begin{frame}[fragile]{Java II}
		\begin{minted}[firstnumber=last]{java}
	public int getAge() {
		return age;
	}
	
	public void setAge(int age) {
		this.age = age;
	}
	
	@Override
	public String toString() {
		return "PersonBean [" + "name=" + name + ", " +	"age=" + age + "]";
	}
	
	@Override
	public int hashCode() {
		final int prime = 31;
		\end{minted}
\end{frame}
	\begin{frame}[fragile]{Java III}
		\begin{minted}[firstnumber=last]{java}
		int result = 1;
		result = prime * result + age;
		result = prime * result + ((name == null) ? 0 : name.hashCode());
		return result;
	}
	
	@Override
	public boolean equals(Object obj) {
		if (this == obj)
			return true;
		if (obj == null)
			return false;
		if (getClass() != obj.getClass())
			return false;
		PersonBean other = (PersonBean) obj;
		if (age != other.age)
		\end{minted}
\end{frame}
	\begin{frame}[fragile]{Java IV}
		\vspace{-2.5cm}
		\begin{minted}[firstnumber=last]{java}
			return false;
		if (name == null) {
			if (other.name != null)
				return false;
		} else if (!name.equals(other.name))
			return false;
		return true;
	}
}
		\end{minted}
\end{frame}
	
	%%%%%%%%%%%%%%%%%%%%%%%%%%%%%%%%%%%%%%%%%%%%%%%%%%%%%%%%%%%%%%%%%%%%%%%

	\begin{frame}{Design}
		Initial design by 3 people with different backgrounds:
		\begin{itemize}
			\item Rob Pike (Concurrency)
			\item Robert Griesemer (Modules)
			\item Ken Thompson (Operating Systems)
		\end{itemize}
		All design decisions had to be agreed upon unanimously.
		Design team later joined by more people at Google.
	\end{frame}
	
	%%%%%%%%%%%%%%%%%%%%%%%%%%%%%%%%%%%%%%%%%%%%%%%%%%%%%%%%%%%%%%%%%%%%%%%
	
	\begin{frame}{Design}
		\begin{itemize}
			\item \textbf{simplicity}
			\item \textbf{simplicity}
			\item \textbf{simplicity}
			\item clean package model for fast compilation
			\item built-in concurrency based on CSP
			\item interfaces instead of inheritance
			\item no radical changes after Go 1.0
		\end{itemize}
	\end{frame}
	
	%%%%%%%%%%%%%%%%%%%%%%%%%%%%%%%%%%%%%%%%%%%%%%%%%%%%%%%%%%%%%%%%%%%%%%%
	
	\begin{frame}
		\begin{center}
			\includegraphics[width=0.85\textwidth]{images/the_general_problem\inv}
		\end{center}
	\end{frame}
	
	%%%%%%%%%%%%%%%%%%%%%%%%%%%%%%%%%%%%%%%%%%%%%%%%%%%%%%%%%%%%%%%%%%%%%%%
	
	\begin{frame}[fragile]{Hello world!}
		\inputminted{go}{code/01_hello.go}
		\\
		\begin{itemize}
			\item The import declaration imports entire packages
			\item All imported names must be qualified
			\item Uppercase names are visible to other packages
			\item Unused imports are compile-time errors!
		\end{itemize}
\end{frame}
	
	%%%%%%%%%%%%%%%%%%%%%%%%%%%%%%%%%%%%%%%%%%%%%%%%%%%%%%%%%%%%%%%%%%%%%%%
	
	\begin{frame}[fragile]{Hello world!}
		Get the \texttt{go} compiler:\\\\
		\begin{minted}{bash}
$ sudo apt-get install golang-go
$ sudo pacman -S go

...or download from https://golang.org/dl
		\end{minted}
		\\\\\\
		Run the code\footnote{The \texttt{go run} command works for single files, not always for projects}:\\\\
		\begin{minted}{bash}
$ go run hello.go
		\end{minted}
\end{frame}
	
	%%%%%%%%%%%%%%%%%%%%%%%%%%%%%%%%%%%%%%%%%%%%%%%%%%%%%%%%%%%%%%%%%%%%%%%
	
	\begin{frame}[fragile]{Basics}
		\begin{center}
			Keywords:
		\end{center}
		\begin{minted}[linenos=false]{go}
break       default     func        interface   select
case        defer       go          map         struct
chan        else        goto        package     switch
const       fallthrough if          range       type
continue    for         import      return      var
		\end{minted}
		\begin{center}
			Constants:
		\end{center}
		\begin{minted}[linenos=false]{go}
true        false       nil         iota
		\end{minted}
\end{frame}

%%%%%%%%%%%%%%%%%%%%%%%%%%%%%%%%%%%%%%%%%%%%%%%%%%%%%%%%%%%%%%%%%%%%%%%

	\begin{frame}[fragile]{Basics}
		\begin{center}
			Functions:
		\end{center}
		\begin{minted}[linenos=false]{go}
new        len          complex     panic
make       cap          real        recover
close      append       imag
           copy 
           delete
		\end{minted}
\end{frame}

%%%%%%%%%%%%%%%%%%%%%%%%%%%%%%%%%%%%%%%%%%%%%%%%%%%%%%%%%%%%%%%%%%%%%%%

	\begin{frame}[fragile]{Basics}
		\begin{center}
			Basic types:
		\end{center}
		\begin{minted}[linenos=false]{go}
int    int8    int16    int32    int64
uint   uint8   uint16   uint32   uint64   uintptr

float32     float64
complex64  complex128

bool   byte    rune     string   error
		\end{minted}
		\\
		\begin{itemize}
			\item \mintinline{go}{int} and \mintinline{go}{uint} are platform-dependent
			\item \mintinline{go}{byte} is the same as \mintinline{go}{uint8}
			\item \mintinline{go}{rune} is the same as \mintinline{go}{uint32}
			\item \mintinline{go}{uintptr} is large enough to hold pointers
			\item \mintinline{go}{error} is a special type for error handling
		\end{itemize}
\end{frame}
	
	%%%%%%%%%%%%%%%%%%%%%%%%%%%%%%%%%%%%%%%%%%%%%%%%%%%%%%%%%%%%%%%%%%%%%%%
	
	\begin{frame}[fragile]{Basics}
		\begin{center}
			Operators:
		\end{center}
		\begin{minted}[linenos=false]{go}
*   /   %   &   &^  <<  >>
+   -   ^   |
==  !=  <   <=  >   >=
&&
||
		\end{minted}
		\\
		\begin{itemize}
			\item only 5 precedence levels!
			\item \mintinline{go}{^} is both bitwise-xor (infix) and bitwise-not (prefix)
			\item \mintinline{go}{&^} is bitwise-andn
		\end{itemize}
\end{frame}
	
	%%%%%%%%%%%%%%%%%%%%%%%%%%%%%%%%%%%%%%%%%%%%%%%%%%%%%%%%%%%%%%%%%%%%%%%
	
	\begin{frame}[fragile]{}
	\end{frame}
\end{document}